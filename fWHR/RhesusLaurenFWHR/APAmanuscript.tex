\documentclass[english,man]{apa6}

\usepackage{amssymb,amsmath}
\usepackage{ifxetex,ifluatex}
\usepackage{fixltx2e} % provides \textsubscript
\ifnum 0\ifxetex 1\fi\ifluatex 1\fi=0 % if pdftex
  \usepackage[T1]{fontenc}
  \usepackage[utf8]{inputenc}
\else % if luatex or xelatex
  \ifxetex
    \usepackage{mathspec}
    \usepackage{xltxtra,xunicode}
  \else
    \usepackage{fontspec}
  \fi
  \defaultfontfeatures{Mapping=tex-text,Scale=MatchLowercase}
  \newcommand{\euro}{€}
\fi
% use upquote if available, for straight quotes in verbatim environments
\IfFileExists{upquote.sty}{\usepackage{upquote}}{}
% use microtype if available
\IfFileExists{microtype.sty}{\usepackage{microtype}}{}

% Table formatting
\usepackage{longtable, booktabs}
\usepackage{lscape}
% \usepackage[counterclockwise]{rotating}   % Landscape page setup for large tables
\usepackage{multirow}		% Table styling
\usepackage{tabularx}		% Control Column width
\usepackage[flushleft]{threeparttable}	% Allows for three part tables with a specified notes section
\usepackage{threeparttablex}            % Lets threeparttable work with longtable

% Create new environments so endfloat can handle them
% \newenvironment{ltable}
%   {\begin{landscape}\begin{center}\begin{threeparttable}}
%   {\end{threeparttable}\end{center}\end{landscape}}

\newenvironment{lltable}
  {\begin{landscape}\begin{center}\begin{ThreePartTable}}
  {\end{ThreePartTable}\end{center}\end{landscape}}

  \usepackage{ifthen} % Only add declarations when endfloat package is loaded
  \ifthenelse{\equal{\string man}{\string man}}{%
   \DeclareDelayedFloatFlavor{ThreePartTable}{table} % Make endfloat play with longtable
   % \DeclareDelayedFloatFlavor{ltable}{table} % Make endfloat play with lscape
   \DeclareDelayedFloatFlavor{lltable}{table} % Make endfloat play with lscape & longtable
  }{}%



% The following enables adjusting longtable caption width to table width
% Solution found at http://golatex.de/longtable-mit-caption-so-breit-wie-die-tabelle-t15767.html
\makeatletter
\newcommand\LastLTentrywidth{1em}
\newlength\longtablewidth
\setlength{\longtablewidth}{1in}
\newcommand\getlongtablewidth{%
 \begingroup
  \ifcsname LT@\roman{LT@tables}\endcsname
  \global\longtablewidth=0pt
  \renewcommand\LT@entry[2]{\global\advance\longtablewidth by ##2\relax\gdef\LastLTentrywidth{##2}}%
  \@nameuse{LT@\roman{LT@tables}}%
  \fi
\endgroup}


\ifxetex
  \usepackage[setpagesize=false, % page size defined by xetex
              unicode=false, % unicode breaks when used with xetex
              xetex]{hyperref}
\else
  \usepackage[unicode=true]{hyperref}
\fi
\hypersetup{breaklinks=true,
            pdfauthor={},
            pdftitle={fWHR in macaques},
            colorlinks=true,
            citecolor=blue,
            urlcolor=blue,
            linkcolor=black,
            pdfborder={0 0 0}}
\urlstyle{same}  % don't use monospace font for urls

\setlength{\parindent}{0pt}
%\setlength{\parskip}{0pt plus 0pt minus 0pt}

\setlength{\emergencystretch}{3em}  % prevent overfull lines

\ifxetex
  \usepackage{polyglossia}
  \setmainlanguage{}
\else
  \usepackage[english]{babel}
\fi

% Manuscript styling
\captionsetup{font=singlespacing,justification=justified}
\usepackage{csquotes}
\usepackage{upgreek}

 % Line numbering
  \usepackage{lineno}
  \linenumbers


\usepackage{tikz} % Variable definition to generate author note

% fix for \tightlist problem in pandoc 1.14
\providecommand{\tightlist}{%
  \setlength{\itemsep}{0pt}\setlength{\parskip}{0pt}}

% Essential manuscript parts
  \title{fWHR in macaques}

  \shorttitle{fWHR in macaques}


  \author{Drew M Altschul\textsuperscript{1}, Lauren Robinson\textsuperscript{1,2}, \& Vanessa Wilson\textsuperscript{1,2}}

  \def\affdep{{"", "", ""}}%
  \def\affcity{{"", "", ""}}%

  \affiliation{
    \vspace{0.5cm}
          \textsuperscript{1} The University of Edinburgh\\
          \textsuperscript{2} fuckin Disney World  }

  \authornote{
    \newcounter{author}
    Add complete departmental affiliations for each author here. Each new
    line herein must be indented, like this line.
    
    Enter author note here.

                      Correspondence concerning this article should be addressed to Drew M Altschul, Postal address. E-mail: \href{mailto:my@email.com}{\nolinkurl{my@email.com}}
                                    }


  \abstract{Enter abstract here. Each new line herein must be indented, like this
line.}
  \keywords{facial width height ratio \\

    \indent Word count: X
  }





\usepackage{amsthm}
\newtheorem{theorem}{Theorem}
\newtheorem{lemma}{Lemma}
\theoremstyle{definition}
\newtheorem{definition}{Definition}
\newtheorem{corollary}{Corollary}
\newtheorem{proposition}{Proposition}
\theoremstyle{definition}
\newtheorem{example}{Example}
\theoremstyle{remark}
\newtheorem*{remark}{Remark}
\begin{document}

\maketitle

\setcounter{secnumdepth}{0}



\section{Introduction}\label{introduction}

In many primate species, the face is used not only as a means of
emotional expression (Calcutt, Rubin, Pokorny \& de Waal, 2017; Parr \&
Waller, 2006; Waller \& Micheletta, 2013) but as a way to communicate
other socially relevant information such as fertility (Rigaill et al.,
2015; Setchell et al. 2006; Dubuc et al., 2009), or health and mate
fitness (Thornhill \& Gangestad, 2006; Kramer \& Ward, 2010; Henderson
et al., 2016; Valentine et al., 2014; Little et al., 2012). Facial
colouration for example has been linked to social status in male
mandrills (Mandrillus sphinx) and drills (Mandrillus leucophaeus), with
stronger facial colouration found in higher ranking males (Setchell et
al. 2008; Marty et al. 2009 ). In rhesus macaques, facial colouration in
males is not linked to dominance status, but is thought to be an
indicator of mate quality (Waitt et al. 2003; Dubuc et al. 2014). It is
possible however that other facial characteristics are linked to rank or
assertive traits in this species. For example, Borgi \& Majolo (2016)
found links between a measure of face width, the facial Width-to-Height
Ratio (fWHR) and dominance style. Across 11 macaque species, fWHR is
higher in both sexes in macaque species with despotic female dominance
styles, such as in rhesus macaques, compared to more socially tolerant
species such as Tonkean macaques. Rhesus macaques are amongst the most
despotic of macaque species, with a strict female dominance hierarchy
and high levels of intra-group aggression (Thierry, 2000). Borgi and
Majolo's findings suggest that face width could be a signal of dominance
that reduces the need for conflict amongst species for whom escalated
conflict could have serious consequences.

Previous work in humans supports links to dominant behaviour. In human
males, aggressiveness and fighting ability have consistently been
associated with a higher fWHR (Haselhuhn, Ormiston \& Wong, 2015; Anderl
et al., 2016; Trebicky et al. 2015; Zilioli et al. 2015; Goetz et al.
2013). Wider male faces are also perceived as being more aggressive
(Alarajih \& Ward; Lefevre \& Lewis 2013; Mileva et al. 2014; Stirrat \&
Perrett, 2010), suggesting that face width is a social cue to aggressive
and assertive behaviour. Similar findings in brown capuchins reveal that
this relationship is not unique to humans (Lefevre et al. 2014; Wilson
et al., 2014). For adults of both sexes, fWHR is related to ratings of
Assertiveness, and furthermore is higher amongst alpha individuals
(Lefevre et al. 2014). Whether the face is a social cue of assertiveness
is still under debate (Wilson et al. under review). Furthermore, it has
been suggested, both in humans (Goetz et al. 2013; Welker, Goetz \&
Carre, 2015) and capuchins (Carré, 2014) that the relationship between
face width and dominant/aggressive behaviour is driven by low social
status, that is, the relationship between fWHR and aggressive behaviour
is only significant amongst low-status individuals. It is possible that,
given that high social status usually comes with conspecifics' social
knowledge of that status, there is no need for high ranking individuals
to physically advertise assertiveness. Contrastingly, for a low status
individual, advertising assertive behaviour via a social cue could be
beneficial to obtaining resources.

Whilst the current literature on facial morphology provides some very
intriguing insights into the social role of physical features, our
interpretations of this data are limited by the choice of study species,
which are primarily human. In the current study, we propose to expand
this research to a wider taxa. Given the links between fWHR and social
tolerance across the Macaca genus (Borgi \& Majolo, 2016), it would be
interesting to explore in more detail whether this ratio is linked
specifically to dominance behaviour in a despotic macaque species. In
the following study we decided to explore the relationship of fWHR in
rhesus macaques to two different measures of dominance, (1) hierarchical
dominance measured using David's scores and (2) personality ratings of
dominance, which assess overall tendencies towards assertive and
aggressive behaviour, rather than social standing. As fWHR correlates
positively with despotism (Borgi \& Majolo, 2016), we predict that (1)
higher fWHRs should be found amongst individuals with higher social
status, similar to findings in capuchins (Lefevre et al., 2014); (2)
fWHR should also be positively related to ratings of dominance; and (3)
if the relationship between fWHR and dominance is driven by low social
status, then we should find a significant relationship amongst
individuals with low social status, but not high social status, as
measured by David's scores.

In addition to examining the fWHR, we will explore a second metric, the
facial Lower-Height-Full-Height ratio (fLHFH). Whilst not well studied,
this metric has previously been used as part of a masculinity index in
human males (Penton-voak et al., 2001). fLHFH has also been associated
with higher ratings of Neuroticism and lower ratings of Attentiveness in
brown capuchins, traits that are both strongly related to vigilance
behaviour (Wilson et al 2014; Morton et al., 2013). In capuchins, Wilson
et al. (2014) proposed that these traits may be conceived of as one form
of social status, prosocial competence, as defined by Lilienfeld et al.
(2012), suggesting that facial height may also play a role in social
cues. Moreover, a recent study suggests that upper face height is an
important component to consider when assessing face width: decreases in
upper face height relative to bizygomatic width led to increased
perceptions of aggressiveness compared with faces with the same width
but increased upper face height (Costa, Lio, Gomez \& Sirigu, 2017).
Thus proportional face height may also be considered as an important
component in understanding facial dominance. To shed more light on the
role of face height in social behaviour, we additionally aimed to
examine the relationship between (1) fLHFH and ratings on six
personality components , labelled Confidence, Openness, Dominance,
Friendliness, Activity and Anxiety, and (2) fLHFH and David's scores.
Based on previous findings in brown capuchins (Wilson et al., 2014) we
predict that (4) ratings of lower Confidence or higher Anxiety were most
likely to be associated with higher fLHFH ratios, and (5) we would not
find a relationship between face height and David's scores.

\_\_I would like to look at this, but we can't with the following
\url{data:__} Given the links between face width and assertive
behaviour, and the link to intraspecific despotism , it would be useful
to explore to what extent social tolerance of different species varies
with variation in fWHRs. If fWHR is dependent on social status, one
would expect to find larger differences in this ratio between alpha and
non-alpha individuals in despotic species over socially tolerant species
.

\section{Methods}\label{methods}

We report how we determined our sample size, all data exclusions (if
any), all manipulations, and all measures in the study.

\subsection{Participants}\label{participants}

\subsection{Material}\label{material}

\subsection{Procedure}\label{procedure}

\subsection{Data analysis}\label{data-analysis}

We used R (3.4.0, R Core Team, 2017) and the R-packages \emph{devtools}
(1.13.3, Wickham \& Chang, 2017), and \emph{papaja} (0.1.0.9492, Aust \&
Barth, 2017) for all our analyses.

\section{Results}\label{results}

apa\_print(m1)

\section{Discussion}\label{discussion}

\newpage

\section{References}\label{references}

\setlength{\parindent}{-0.5in} \setlength{\leftskip}{0.5in}

\hypertarget{refs}{}
\hypertarget{ref-R-papaja}{}
Aust, F., \& Barth, M. (2017). \emph{papaja: Create APA manuscripts with
R Markdown}. Retrieved from \url{https://github.com/crsh/papaja}

\hypertarget{ref-R-base}{}
R Core Team. (2017). \emph{R: A language and environment for statistical
computing}. Vienna, Austria: R Foundation for Statistical Computing.
Retrieved from \url{https://www.R-project.org/}

\hypertarget{ref-R-devtools}{}
Wickham, H., \& Chang, W. (2017). \emph{Devtools: Tools to make
developing r packages easier}. Retrieved from
\url{https://CRAN.R-project.org/package=devtools}






\end{document}
